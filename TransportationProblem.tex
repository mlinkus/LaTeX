\documentclass[12pt,letterpaper]{article}
\usepackage{amssymb, amsthm}
\usepackage{mathtools}

\author{M. Linkus\\
St. John's University}
\title{Transportation Problem
\\Shipping Wood to Market}
\date{\today}

\newtheoremstyle{named}{}{}{\itshape}{}{\bfseries}{.}{.5em}{\thmnote{#3}#1}
\theoremstyle{named}
\newtheorem*{theorem}{}

\begin{document}
\maketitle
\begin{abstract}
	A very brief overview of Operations Research and the Transportation Problem for the purpose of
	examining a particular case study.  The case study revolves around optimization of shipping costs
	for	a lumber manufacturer which must get wood from different sources to various markets.  Since
	the manufacturer is considering switching shipments from rail to sea, the objective is to determine
	the most cost-effective options.
\end{abstract}
\tableofcontents
\pagebreak

\section{Questions}
\begin{theorem}[Questions]
	(i.) What question(s) would you like the answer with your project?
	(ii.) What question(s) can you answer now?
	(iii.) What direction are you taking next?
\end{theorem}

%

\section{Liouville's Theorem}
\begin{theorem}[Statement]
	The only bounded entire functions are constant.
\end{theorem}
\begin{proof}[Proof.]
Let $f(z)$ be a bounded, entire function and let $z$ be a fixed point of $\mathbb{C}$.

Let $C = \{w \in \mathbb{C}: |w| = R \}$ where $R \gg 1$ and $R \gg |z|$.

By Cauchy's Integral Formula,
$$f(z) = \frac{1}{2\pi i} \oint_C \frac{f(w)}{w-z} ~dw $$
and
$$f(0) = \frac{1}{2\pi i} \oint_C \frac{f(w)}{w} ~dw $$
Therefore
$$|f(z) - f(0)| = \left|\frac{1}{2\pi i} \oint_C \left(\frac{f(w)}{w-z} - \frac{f(w)}{w}\right) ~dw \right|$$

$$|f(z) - f(0)| \leq \frac{1}{2\pi} \oint_C \frac{|f(w)| \cdot |z|}{|w (w-z)|} ~|dw| $$
Since $f$ is bounded, $|f(z)| \leq M < \infty ~\forall ~z \in \mathbb{C}$, therefore
$$|f(z) - f(0)| \leq \frac{M \cdot |z|}{2\pi} \oint_C \frac{1}{R (R-|z|)} ~|dw| $$
Since $2\pi R$ is the length of $C$ and $R (R-|z|)$ is constant with respect to $w$,
$$|f(z) - f(0)| \leq \frac{M \cdot |z|}{2\pi} \cdot \frac{2\pi R}{R (R-|z|)} = \frac{M \cdot |z|}{R-|z|} \to 0 \text{ as } R \to \infty$$
Therefore, $f(z) = f(0)$.  Since $z$ was any point of $\mathbb{C}$, $f$ is constant.
\end{proof}

%

\section{Riemann's Theorem on Removable Singularities}
\begin{theorem}[Statement]
	If $f$ is analytic and bounded in some punctured neighborhood of $z_0$, then $\lim_{z \to z_0} f(z)$
	exists.  If $f(z_0)$ is defined as this limit, then $f$ becomes analytic at $z_0$.
\end{theorem}
\begin{proof}[Proof.]
Define $g(z) =
	\begin{cases}
		(z - z_0 )^2 f(z) & \text{if } z \neq z_0 \\
		0       & \text{if } z = z_0
	\end{cases}
$
\\ We show $g$ is analytic in a neighborhood of $z_0$.  $g$ is differentiable at $z_0$ because
$$ g'(z_0) = \lim_{z \to z_0} \frac{g(z) - g(z_0)}{z - z_0} = \lim_{z \to z_0} \frac{(z-z_0)^2 f(z) - 0}{z - z_0} = \lim_{z \to z_0} (z-z_0) f(z) = 0 $$
The last step is valid since $|f(z)|$ is bounded.  If $z \neq z_0$,
$$ g'(z) = 2(z-z_0)f(z) + (z-z_0)^2 f'(z) $$
So $g$ is analytic in a neighborhood of $z_0$.  Therefore, it coincides with its Taylor Series centered at $z_0$, that is,
$$ g(z) = a_0 + a_1 (z-z_0) + a_2 (z-z_0)^2 + a_3 (z-z_0)^3 + \cdots $$
But, $a_0 = g(z_0) = 0$ and $a_1 = g'(z_0) = 0$, therefore
$$ g(z) = a_2 (z-z_0)^2 + a_3 (z-z_0)^3 + \cdots $$
$$ f(z) = a_2 + a_3(z-z_0) + \cdots $$
Set $f(z_0) = a_2$.  Since the power series is valid in a neighborhood of $z_0$ and $f(z)$ is bounded there,
$f$ is analytic at $z_0$.
\end{proof}

%

\section{Rouch\'{e}'s Theorem}
\begin{theorem}[Statement]
	Let $f$ and $g$ be analytic inside and on a simple, closed curve $C$ and suppose that
	$|g(z)| < |f(z)|$ when $z$ is on $C$.  Then $f(z)$ and $f(z) + g(z)$ have the same number
	of zeroes inside $C$.
\end{theorem}
\begin{proof}[Proof.]
Let $N$ be the number of zeroes of $f(z) + g(z)$ inside $C$.
\\By the Argument Principle,
$$ N = \frac{1}{2\pi i} \oint_C \frac{f' + g'}{f + g} ~dz $$
$$ N = \frac{1}{2\pi i} \oint_C \frac{f'}{f} ~dz + \frac{1}{2\pi i} \oint_C \left(\frac{f' + g'}{f + g} - \frac{f'}{f} \right) ~dz$$
Let $I = \oint_C \left(\frac{f' + g'}{f + g} - \frac{f'}{f} \right) ~dz$
\\If we can show that $I = 0$, then we are done.  Find a common denominator and
$$ I = \oint_C \frac{g'f - gf'}{f(f+g)} ~dz $$
$$ I = \oint_C \frac{\frac{g'f - gf'}{f^2}}{1+\frac{g}{f}} ~dz $$
Let $\varphi = \frac{g}{f}$, then
$$ I = \oint_C \frac{\varphi '}{1 + \varphi} ~dz $$
Since $|\varphi| < 1$ on $C$,
$$ I = \oint_C \varphi '(1 - \varphi + \varphi^2 - \varphi^3 + \cdots) ~dz $$
$$ I = \oint_C \varphi ' ~dz - \oint_C \varphi '\varphi ~dz +\oint_C \varphi '\varphi^2 ~dz - \oint_C \varphi '\varphi^3 ~dz + \cdots $$
$$ I = \oint_C (\varphi)' ~dz - \frac{1}{2}\oint_C (\varphi^2)' ~dz + \frac{1}{3}\oint_C (\varphi^3)' ~dz - \frac{1}{4}\oint_C (\varphi^4)' ~dz + \cdots $$
We can see each integral is of a derivative and therefore zero, thus $I = 0$.
\end{proof}

%

\section{Open Mapping Theorem}
\begin{theorem}[Statement]
	If a nonconstant function $f$ is analytic on a connected, open set $S$, then the image of
	$S$ under $w = f(z)$ is an open set.
\end{theorem}
\begin{proof}[Proof.]
Let $f(z)$ be a nonconstant analytic function on a connected, open set $S$.
\\ Let $z_{0} \in S$ and set $w_{0} = f(z_{0})$.
\\ Define $g(z) = f(z) - w_{0}$, a translation.  $g(z)$ is nonconstant, analytic, and $g(z_{0}) = 0$, which must
be an isolated zero of $g$.
\\ Therefore, there is an $r > 0$ such that the disk $\gamma = \{ |z - z_{0}| \leq r \}$ contains no other zeroes of $g$.
\\ Let $\delta = \min\limits_{|z - z_{0}| = r} |g(z)|$.
Let $\Delta$ be the open disk  $\{ |w - w_{0}| < \delta \}$.
\\ Then for any $z$ on the circle $C = \{z: |z - z_{0}| = r\}$ we have $|w - w_{0}| < |g(z)|$ for some fixed $w \in \Delta$.
\\ By Rouch\'{e}'s Theorem, $g(z)$ and $g(z) - (w - w_{0})$ have the same number of zeroes inside the disk $C$.
\\ Since $g$ has at least one zero inside $C$, $g(z) - w + w_{0} = f(z) - w_{0} - w + w_{0} = f(z) - w$
has at least one zero inside $C$.
\\ So $w = f(z)$ for at least one $z$ inside $C$.  In other words,
$w$ is in the image under $f$ of the disk bounded by $\gamma$.
Therefore, every point $w_{0}$ in the image of $f$ has a neighborhood contained in the image of $f$.
\end{proof}

%

\begin{thebibliography}{44}
\bibitem{Hillier}
Frederick S Hillier, Gerald J Lieberman, \emph{Introduction to Operations Research}, 9e, 2010, ISBN 0073376299
\end{thebibliography}

\end{document}