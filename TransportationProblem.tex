\documentclass[12pt,letterpaper]{article}
\usepackage{amsmath, amssymb, amsthm}
\usepackage{mathtools}
\usepackage{graphicx}
\usepackage{hyperref}
\usepackage[margin=1.0in]{geometry}

\author{M. Linkus\\
St. John's University}
\title{The Transportation Problem
\\Shipping Wood to Market}
\date{\today}

\begin{document}
\maketitle
\begin{abstract}
	This paper introduces the basic concepts of Operations Research, in particular the Transportation Problem, for
the purpose of examining a particular case study about a lumber manfacturer.  The lumber manufacturer is trying to
determine the most cost-effective way to ship goods from various supplies to different demand points and if	it
would cheaper to do so using railroads or boats.
\end{abstract}
\tableofcontents
\pagebreak

\section{Introduction}

 Operations Research is a branch of mathematics concerned primarily with the applications of analysis to improve
decision making. Often, this involves optimization, determining the maximum (profit, performance, etc.) or minimum
(cost, risk, loss) of some objective. While the ideas of optimization and preliminary concepts of Operations Research
were used in the 1800s, their origins trace back to military planning in World War II.  They have since been applied to
fields all across business, industry, and society.

 One problem within Operations Research is known as the Transportation Problem, which is concerned with the optimal
transport of goods, people, resources, etc. This could include finding a minimum cost or a minimum shipment time. It
can also be applied to other problems which are not related to transportation, such as production scheduling.
Typically, problems are solved using linear programming or linear optimization techniques, which is how the case study
in this paper will be approached.

 Linear programming is a set of techniques to optimize a linear function of $n$ variables which is subject to linear
equality or inequality constraints.  One can imagine the constraints as the boundaries of a polytope, typically one
of $n$ dimensions.  (A polytope is a geometric object with flat sides in any general number of dimensions.)  This makes
visualizing all but the basic problems difficult.  Theory tells us that the optimal solution must reside somewhere
inside or along the edges of this polyhedron.  The problem can be expressed as follows:

$$
\begin{aligned}
& \text{Maximize:}
& & c_{11}x_{11} + c_{12}x_{12} + \cdots + c_{1n}x_{1n} \\
&&& c_{21}x_{21} + c_{22}x_{22} + \cdots + c_{2n}x_{2n} \\
&&& \vdots \\
&&& c_{m1}x_{m1} + c_{m2}x_{m2} + \cdots + c_{mn}x_{mn} \\
& \text{subject to:}
& & a_{11}x_1 + a_{12}x_2 + \cdots + a_{1n} \leq b_1 \\
&&& a_{21}x_1 + a_{22}x_2 + \cdots + a_{2n} \leq b_2 \\
&&& \vdots \\
&&& a_{m1}x_1 + a_{m2}x_2 + \cdots + a_{mn} \leq b_m \\
& \text{and}
& & x_{ij} \geq 0 \text{ for all } i, j \\
\end{aligned}
$$

Where $c_{ij}$ and $a_{ij}$ are known constants and the variables $x_{ij}$ are to be determined.  The function to be
maximized is known as the objective function.  Problems can be converted and reformulated between minimization and
maximization as needed, with the inequalities reversed.  The constraints can also be equalities, as they will be in the
case study in this paper.  To solve transportation problems, the equations and inequalities are abstracted as matrices
and matrix methods are applied to find the optimal solutions, either manually or using software.


\section{Case Study}
\subsection{Overview}
In this case study, a lumber company, Alabama Atlantic, seeks to ship large quantities of wood from three sources to
five market destinations.  The annual availability of wood at Sources 1, 2, and 3 is 15, 20, and 15 million board feet,
respectively.  The annual demand for wood at Markets 1 through 5 is 11, 12, 9, 10, and 8 million board feet,
respectively.

In the past, the company has shipped the wood by railroad, but increasing costs have led Alabama Atlantic to consider
shipment by sea.  So, the minimal cost of each option must be considered, and then the minimal cost combining the two
options to determine the company's best plan of action.

\subsection{Transportation By Rail}
We begin by examining the first option, shipping exclusively by rail.  The costs are summarized in the following table,
which has been input into Microsoft Excel:

\begin{center}
\includegraphics[scale=.50]{railcosts.png}\\
Table 1: Cost of Shipment by Rail
\end{center}

The costs from each source to each destination market are known, as well as the supply from each source and the demand
at each destination market.

Let $x_{ij}$ denote the number of units of wood, in millions of board feet, to ship from source $i$ to market $j$.  The
numbers in Table 1 are our costs, or $c_{ij}$ variables.

The optimization problem is modeled as follows:

$$
\begin{aligned}
& \text{Minimize:}
& & 61x_{11} + 72x_{12} + 45x_{13} + 55x_{14} + 66x_{15} + \\
&&& 69x_{21} + 78x_{22} + 60x_{23} + 49x_{24} + 56x_{25} + \\
&&& 59x_{31} + 66x_{32} + 63x_{33} + 61x_{34} + 47x_{35} \\
& \text{subject to:}
& & x_{11} + x_{12} + x_{13} + x_{14} + x_{15} = 15 \\
&&& x_{21} + x_{22} + x_{23} + x_{24} + x_{25} = 20 \\
&&& x_{31} + x_{32} + x_{33} + x_{34} + x_{35} = 15 \\
&&& x_{11} + x_{21} + x_{31} = 11 \\
&&& x_{12} + x_{22} + x_{32} = 12 \\
&&& x_{13} + x_{23} + x_{33} = 9 \\
&&& x_{14} + x_{24} + x_{34} = 10 \\
&&& x_{15} + x_{25} + x_{35} = 8 \\
& \text{and}
& & x_{ij} \geq 0 \text{ for all } i, j \\
\end{aligned}
$$

The objective function and the constraints are all linear in $x_{ij}$. The first three constraints are the supply
constraints; units shipped from each source cannot exceed the supply there. The next five constraints are the demand
constraints, as the shipments to each destination must satisfy the demands there. The last constraint requires that the
$x_{ij}$ are non-negative, because it does not make sense to ship negative amounts.

The function to be optimized is one of 15 variables, so it would not be very practical to solve it using written matrix
methods or tables.  Instead, this problem will be solved using software, in this case the solver which is part of
Microsoft Excel.  Doing so yields the following results shown in the following table:

\begin{center}
\includegraphics[scale=.50]{railsolution.png}\\
Table 2: Optimal Solution for Shipment by Rail
\end{center}

The values from Table 2 for $x_{ij}$ produce a total minimal cost of \$2,816,000 for shipping exclusively by rail.

\subsection{Transportation By Sea}
Next, consider the minimal cost of shipping the wood entirely by sea.  The following costs of shipment are given
and summarized in the following table which has been input into Microsoft Excel:

\begin{center}
\includegraphics[scale=.50]{seacosts.png}\\
Table 3: Cost of Shipment by Sea
\end{center}

There is no route by sea available from source 1 to market 4 or from source 3 to market 1.  To compensate for this,
$x_{14}$ and $x_{31}$ can be ommitted entirely from the problem formulation, or the costs can be considered to be
arbitrarily large.

In the given problem formulation, the costs for those routes are considered to be arbitrarily large.  The supply and
demand constrains are the same and the problem is formulated as follows:

$$
\begin{aligned}
& \text{Minimize:}
& & 58.5x_{11} + 68.3x_{12} + 47.8x_{13} + \;\;\,Mx_{14} + 63.5x_{15} + \\
&&& 65.3x_{21} + 74.8x_{22} + 55.0x_{23} + 49.0x_{24} + 57.5x_{25} + \\
&&& \;\;\,Mx_{31} + 61.3x_{32} + 63.5x_{33} + 58.8x_{34} + 50.0x_{35} \\
& \text{subject to:}
& & x_{11} + x_{12} + x_{13} + x_{14} + x_{15} = 15 \\
&&& x_{21} + x_{22} + x_{23} + x_{24} + x_{25} = 20 \\
&&& x_{31} + x_{32} + x_{33} + x_{34} + x_{35} = 15 \\
&&& x_{11} + x_{21} + x_{31} = 11 \\
&&& x_{12} + x_{22} + x_{32} = 12 \\
&&& x_{13} + x_{23} + x_{33} = 9 \\
&&& x_{14} + x_{24} + x_{34} = 10 \\
&&& x_{15} + x_{25} + x_{35} = 8 \\
& \text{and}
& & x_{ij} \geq 0 \text{ for all } i, j \\
\end{aligned}
$$

where $M$ is arbitrarily large.  Since $M$ is arbitrarily large, it is easily inferred that $x_{14}$ and $x_{31}$ will
both be zero in the optimal solution.  Intuitively, it is not readily apparent if the optimal solution for minimizing
this function will result in a lower cost of shipping than from shipping exclusively by rail.  This is because the
costs of shipping are lower from some sources to some destinations but higher for others.  The optimal solution here
will necessarily be different because the shipping plan by rail given in Table 2 included shipping three units from
source 3 to market 1, where no sea route exists.

Running the solver in Excel again, the following optimal solution is found and shown in the following table:

\begin{center}
\includegraphics[scale=.50]{seasolution.png}\\
Table 4: Optimal Solution for Shipment by Sea
\end{center}

The values from Table 4 for $x_{ij}$ produce a total minimal cost of \$2,770,800 for shipping exclusively by sea. Since
the minimized cost by rail was found to be \$2,816,000, Alabama Atlantic can save \$45,200 by switching from rail to
sea shipping.  However, there is one other option to consider.

\subsection{Transportation By Combination}
The third option is a combination of sea and rail shipping, using the cheapest option available between each source
and destination.  Intuitively, this will be the lowest cost option since it uses the minimum of two costs for each
source and desination.  Finding the optimal solution will show exactly how much cheaper the projected shipping costs
will be using the best combination of two options.

The costs in the following table are the minimums of the costs by rail or sea, as formatted in Excel:

\begin{center}
\includegraphics[scale=.50]{combocosts.png}\\
Table 5: Cost of Shipment by Combination
\end{center}

The formulation of the problem is much the same, except with different coefficients for cost for some of the $x_{ij}$.
The constraints are the same and problem is formulated once again as shown:

$$
\begin{aligned}
& \text{Minimize:}
& & 58.5x_{11} + 68.3x_{12} + 45x_{13} + 55.0x_{14} + 63.5x_{15} + \\
&&& 65.3x_{21} + 74.8x_{22} + 55x_{23} + 49.0x_{24} + 56.0x_{25} + \\
&&& 59.0x_{31} + 61.3x_{32} + 63x_{33} + 58.8x_{34} + 47.0x_{35} \\
& \text{subject to:}
& & x_{11} + x_{12} + x_{13} + x_{14} + x_{15} = 15 \\
&&& x_{21} + x_{22} + x_{23} + x_{24} + x_{25} = 20 \\
&&& x_{31} + x_{32} + x_{33} + x_{34} + x_{35} = 15 \\
&&& x_{11} + x_{21} + x_{31} = 11 \\
&&& x_{12} + x_{22} + x_{32} = 12 \\
&&& x_{13} + x_{23} + x_{33} = 9 \\
&&& x_{14} + x_{24} + x_{34} = 10 \\
&&& x_{15} + x_{25} + x_{35} = 8 \\
& \text{and}
& & x_{ij} \geq 0 \text{ for all } i, j \\
\end{aligned}
$$

The optimal solution is found once again using Excel, and shown in the following table:

\begin{center}
\includegraphics[scale=.50]{combosolution.png}\\
Table 6: Optimal Solution for Shipment by Combination
\end{center}

Curiously, this is the same optimal solution from Table 4, for shipment by sea.  However, it was not readily apparent
from the problem formulation that this would be the case.

Substituting the values from Table 6 for $x_{ij}$ in of the cost function yields a minimal cost of \$2,729,100.  This
is \$86,900 cheaper than the cost of shipping exclusively by rail and \$41,700 cheaper than the cost of shipping
exclusively by sea.  The cost summaries are compiled in the next section.

\subsection{Summary}
By finding the optimal solutions of all three options, the following results are obtained:

$$
\begin{aligned}
& \textbf{Option}
& & \textbf{Minimal Cost} \\
& \text{Rail Only}
& & \text{\$2,816,000} \\
& \text{Sea Only}
& & \text{\$2,770,800 (\$45,200 less than rail)} \\
& \text{Combination:}
& & \text{\$2,729,100 (\$86,900 less than rail, \$41,700 less than sea)} \\
\end{aligned}
$$

Clearly, Alabama Atlantic stands to save a significant amount of money, up to about 3.08\% of costs on shipping, by
incorporating sea transport into its shipping plan.


\begin{thebibliography}{44}
\bibitem{Hillier}
Frederick S. Hillier \& Gerald J. Lieberman, \emph{Introduction to Operations Research}, 9e, 2010, ISBN 0073376299

\bibitem{Wikipedia}
Various Authors, Wikipedia \\
\emph{Operations Research}, \href{http://en.wikipedia.org/wiki/Operations_research}{http://en.wikipedia.org/wiki/Operations\_research} \\
\emph{Linear Programming}, \href{http://en.wikipedia.org/wiki/Linear_programming}{http://en.wikipedia.org/wiki/Linear\_programming},
\end{thebibliography}

\end{document}